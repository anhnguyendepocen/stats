\documentclass[10pt]{article}

\usepackage{graphicx}
\usepackage{amssymb}
\usepackage{amsmath}
\usepackage{ifthen}
\usepackage{epsfig}
\usepackage{fancyvrb}
\usepackage{url}
\usepackage{fancyhdr}

\usepackage{amsmath}
\usepackage{mathspec}
\usepackage{microtype}
\setmainfont[
    Numbers={Monospaced,OldStyle},
    BoldFont={Guardian TextEgyp Medium},
  ]{Guardian TextEgyp}\linespread{1.2}
\setsansfont[BoldFont={Helvetica Neue Medium}]{Helvetica Neue}\linespread{1.2}
\setmonofont{Monaco}\linespread{1.2}
\setmathrm{Palatino}
\setmathfont(Digits,Latin,Greek){Palatino}

\tolerance=5000

\usepackage[margin=1.0in,includeheadfoot]{geometry}

\setlength{\skip\footins}{10mm}      % obsessing about footnote spacing

\usepackage[compact,noindentafter]{titlesec}
\titleformat*{\section}{\sffamily\large\bfseries}
\titlespacing{\section}{0pt}{2ex}{1ex}
\titleformat*{\subsection}{\sffamily\normalsize\bfseries}
\titlespacing{\subsection}{0pt}{2ex}{0ex}

\usepackage[iso]{datetime}
\usdate

% ------- CUSTOM TITLE FORMAT -------
%
\makeatletter
\renewcommand{\maketitle}{
\begin{flushleft}          % right align
{\Large\sffamily\bfseries\@title}   % increase the font size of the title
\vspace{2ex}\\            % vertical space between the title and author name
{\normalsize\sffamily\@author}           % author name
\vspace{0ex}\\             % vertical space between author name and date
\normalsize\sffamily\@date                     % date
\vspace{3ex}              % vertical space between the author block and abstract
\end{flushleft}
}
% -----------------------------------


\title{Assignment 1---Introduction to Statistics Using R}
\author{Paul Gribble}
\date{Winter, 2017}


\begin{document}

\maketitle

\thispagestyle{empty}

{\flushleft \sffamily * Due Friday Jan 27}

\section{Hypothesis Testing}

You are Dean of Science at Western (congratulations!). You have just
met with one of the new Assistant Professors in the Department of
Mathematics for his annual performance evaluation. Professor X claims
that his salary is much too low and he wants you to give him a
raise. He claims he is particularly good at teaching, and so he
deserves a raise. As evidence of this, he provides a random sample of
his teaching ratings. In his previous year, Professor X taught Section
I of Calculus, and another Professor (with a higher salary) taught
Section II. The students in Section I achieved the highest average
grades on the final exam, which was common to both sections. Professor
X provides you with the final grades of all 50 students in each of the
two calculus
sections~\footnote{\url{http://www.gribblelab.org/stats/data/calculus.csv}}.

Table~\ref{calctable} shows mean grades for each of the two sections.

\begin{table}[h]
  \begin{center}
    \begin{tabular}{|l|cc|}
      \hline
      Calculus Section &I &II\\
      \hline
      Mean Final Grade &78.9 &75.3\\
      \hline
    \end{tabular}
    \caption{Final Calculus Grades}
    \label{calctable}
  \end{center}
\end{table}

There is no doubt that the average final grade in Section I is higher
than it is for Section II. You are a natural skeptic however.

\begin{enumerate}

\item Articulate a null hypothesis in plain language.

\item Calculate the probability of observing differences between
  grades this large, under the null hypothesis. Show your work. Be
  sure to consider whether this ought to be a one-tailed or a
  two-tailed t-test.

\item What assumptions do you have to make when performing your
  analysis? If possible, test those assumptions. (hint:
  \texttt{bartlett.test()} will test homogeneity of variances, and
  \texttt{shapiro.test()} will test for normality.

\item What will you conclude about Professor X's claim? Will you give
  him a raise? Why or why not?

\item For your future reference as Dean, determine the minimum
  difference between mean grades that you would need to see in order
  to reject the null hypothesis. Assume everything else stays the same
  (variances, sample sizes, etc). Show your work. Hint: compute the
  value of t needed for p=.05 given the degrees of freedom in the data
  (use the R function \texttt{qt()}, then solve for the difference
  between means that's required to produce that t value (using the
  equation for t).


\end{enumerate}


\end{document}


%%% Local Variables:
%%% mode: latex
%%% TeX-master: t
%%% End:
