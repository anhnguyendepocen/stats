\documentclass[10pt]{article}

\usepackage{graphicx}
\usepackage{amssymb}
\usepackage{amsmath}
\usepackage{ifthen}
\usepackage{epsfig}
\usepackage{fancyvrb}
\usepackage{url}
\usepackage{fancyhdr}

\usepackage{amsmath}
\usepackage{mathspec}
\usepackage{microtype}
\setmainfont[
    Numbers={Monospaced,OldStyle},
    BoldFont={Guardian TextEgyp Medium},
  ]{Guardian TextEgyp}\linespread{1.2}
\setsansfont[BoldFont={Helvetica Neue Medium}]{Helvetica Neue}\linespread{1.2}
\setmonofont[Scale=0.90]{Menlo}\linespread{1.2}
\setmathrm{Palatino}
\setmathfont(Digits,Latin,Greek){Palatino}

\tolerance=5000

\usepackage[margin=1.25in,includeheadfoot]{geometry}

\setlength{\skip\footins}{10mm}      % obsessing about footnote spacing
\setlength{\parskip}{1ex}            % paragraph spacing
\setlength{\parindent}{0ex}          % paragraph indentation

\usepackage[compact,noindentafter]{titlesec}
\titleformat*{\section}{\sffamily\large\bfseries}
\titlespacing{\section}{0pt}{2ex}{1ex}
\titleformat*{\subsection}{\sffamily\normalsize\bfseries}
\titlespacing{\subsection}{0pt}{2ex}{0ex}

% LISTS
\usepackage{enumitem}
\setenumerate{nolistsep,itemsep=1.0ex,parsep=0.0ex,leftmargin=*}
\setitemize{nolistsep,itemsep=1.0ex,parsep=0.0ex,leftmargin=*}

\usepackage[iso]{datetime}
\usdate

% ------- CUSTOM TITLE FORMAT -------
%
\makeatletter
\renewcommand{\maketitle}{
\begin{flushleft}          % right align
{\Large\sffamily\bfseries\@title}   % increase the font size of the title
\vspace{3ex}\\            % vertical space between the title and author name
{\normalsize\sffamily\@author}           % author name
\vspace{0ex}\\             % vertical space between author name and date
\normalsize\sffamily\@date                     % date
\vspace{5ex}              % vertical space between the author block and abstract
\end{flushleft}
}
% -----------------------------------

% DEFINE SOME COLORS
\usepackage{color}
\definecolor{light-blue}{RGB}{37,128,162}
\definecolor{light-gray}{gray}{0.975}
\definecolor{dark-gray}{gray}{0.50}

% CODE LISTINGS
\usepackage{listings}
\lstset{columns=fullflexible,
  basicstyle=\ttfamily\small,
  backgroundcolor=\color{light-gray},
  frame=lines, frame=lrtb, framesep=3pt, rulecolor=\color{dark-gray},
  aboveskip=10pt,
  belowskip=15pt,
  breaklines=true,
  showstringspaces=false,
  keepspaces=true,
}


\title{Assignment 3---Introduction to Statistics Using R}
\author{Paul Gribble}
\date{Winter, 2017}


\begin{document}

\maketitle

\thispagestyle{empty}

{\flushleft \sffamily * Due Sunday Feb 19}

\section*{Snakes on a Plane}

A psychologist is interested in three types of therapy for modifying
snake phobia. She suspects that one type of therapy may not be best
for everyone---rather the best type of therapy may depend on the
severity of the phobia. Undergraduate students in an intro psych
course are given the Fear Schedule Survey (FSS) to screen out subjects
showing no fear of snakes. Those displaying some degree of phobia are
classified as mildly, moderately, or severely phobic. One third of
subjects within each level of severity are then assigned to a
treatment condition: systematic desensitization, implosive therapy, or
insight therapy. The following data are obtained following therapy,
using the Behavioral Avoidance Test (higher scores are better, meaning
less snake phobia)
\footnote{\url{http://www.gribblelab.org/stats/data/snakedata.csv}}.

Your task is to analyze these data and to answer any questions you
believe would be of theoretical interest. Don't feel compelled to
perform an analysis just because it is possible statistically. Rather
justify the use of each test. Don't forget to test assumptions of
ANOVA, and to justify your choices about how (or if) to control for
Type-I error in any follow-up tests you choose to perform. Use graphical plots
where appropriate to help tell the story. Finally, make sure to include one or more
concluding statements.

\begin{table}[h]
  \begin{center}
    \begin{tabular}{|rrr|rrr|rrr|}
      \hline
      \multicolumn{3}{|c|}{Desensitization} &\multicolumn{3}{|c|}{Implosion} &\multicolumn{3}{|c|}{Insight} \\
      \hline
      Mild & Moderate & Severe & Mild & Moderate & Severe & Mild & Moderate & Severe \\
      14 & 15 & 12 & 10 & 12 & 10 &  8 & 9  &  6 \\
      17 & 11 & 10 & 16 & 14 &  3 & 10 & 6  & 10 \\
      10 & 12 & 10 & 19 & 10 &  6 & 12 & 7  &  8 \\
      13 & 10 &  9 & 20 & 11 &  8 & 14 & 12 &  9 \\
      12 &  9 & 11 & 19 & 13 &  2 & 11 & 11 &  7 \\
      \hline
      \end{tabular}
      \caption{BAT scores following therapy for snake phobia (higher scores are better, indicating less phobia)}
      \label{ass5data}
      \end{center}
\end{table}

\end{document}

