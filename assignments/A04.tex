\documentclass[10pt]{article}

\usepackage{graphicx}
\usepackage{amssymb}
\usepackage{amsmath}
\usepackage{ifthen}
\usepackage{epsfig}
\usepackage{fancyvrb}
\usepackage{url}
\usepackage{fancyhdr}

\usepackage{amsmath}
\usepackage{mathspec}
\usepackage{microtype}
\setmainfont[
    Numbers={Monospaced,OldStyle},
    BoldFont={Guardian TextEgyp Medium},
  ]{Guardian TextEgyp}\linespread{1.2}
\setsansfont[BoldFont={Helvetica Neue Medium}]{Helvetica Neue}\linespread{1.2}
\setmonofont[Scale=0.90]{Menlo}\linespread{1.2}
\setmathrm{Palatino}
\setmathfont(Digits,Latin,Greek){Palatino}

\tolerance=5000

\usepackage[margin=1.25in,includeheadfoot]{geometry}

\setlength{\skip\footins}{10mm}      % obsessing about footnote spacing
\setlength{\parskip}{1ex}            % paragraph spacing
\setlength{\parindent}{0ex}          % paragraph indentation

\usepackage[compact,noindentafter]{titlesec}
\titleformat*{\section}{\sffamily\large\bfseries}
\titlespacing{\section}{0pt}{2ex}{1ex}
\titleformat*{\subsection}{\sffamily\normalsize\bfseries}
\titlespacing{\subsection}{0pt}{2ex}{0ex}

% LISTS
\usepackage{enumitem}
\setenumerate{nolistsep,itemsep=1.0ex,parsep=0.0ex,leftmargin=*}
\setitemize{nolistsep,itemsep=1.0ex,parsep=0.0ex,leftmargin=*}

\usepackage[iso]{datetime}
\usdate

% ------- CUSTOM TITLE FORMAT -------
%
\makeatletter
\renewcommand{\maketitle}{
\begin{flushleft}          % right align
{\Large\sffamily\bfseries\@title}   % increase the font size of the title
\vspace{3ex}\\            % vertical space between the title and author name
{\normalsize\sffamily\@author}           % author name
\vspace{0ex}\\             % vertical space between author name and date
\normalsize\sffamily\@date                     % date
\vspace{5ex}              % vertical space between the author block and abstract
\end{flushleft}
}
% -----------------------------------

% DEFINE SOME COLORS
\usepackage{color}
\definecolor{light-blue}{RGB}{37,128,162}
\definecolor{light-gray}{gray}{0.975}
\definecolor{dark-gray}{gray}{0.50}

% CODE LISTINGS
\usepackage{listings}
\lstset{columns=fullflexible,
  basicstyle=\ttfamily\small,
  backgroundcolor=\color{light-gray},
  frame=lines, frame=lrtb, framesep=3pt, rulecolor=\color{dark-gray},
  aboveskip=10pt,
  belowskip=15pt,
  breaklines=true,
  showstringspaces=false,
  keepspaces=true,
}


\title{Assignment 4---Introduction to Statistics Using R}
\author{Paul Gribble}
\date{Winter, 2017}


\begin{document}
\maketitle

\thispagestyle{empty}

{\flushleft \sffamily * Due Friday March 3}

\section{ANCOVA}

You are Dean of Science at UWO. You have just met with one of the new
assistant professors in the Department of Math, for his annual
performance evaluation. Professor X claims that his salary is much too
low and he wants you to give him a raise. He claims he is particularly
good at teaching and thus he deserves a raise. As evidence of this, he
says that among the three sections of calculus taught in first year,
his students (Section I) get the highest grades on the Calculus final
exam, which is common to all three sections (see Table~\ref{calctable}
below). You ask the Chair of Math to collect some data for you. She
gives you final exam scores for a random sample of 50 students chosen
from each of the three sections of Calculus
101~\footnote{\url{http://www.gribblelab.org/stats/data/calculus2.csv}}.

\textbf{Question 1.1}: Conduct a standard one-way ANOVA to assess
whether the three sections differ in terms of final exam scores. What
would you conclude based on the ANOVA?

The Chair of the Math Department calls you the next day and tells you
that she also has available high-school mathematics grades for each of
the students in the three sections of Calculus 101. She advises that
you might want to ``take this information into consideration'' when
comparing final grades in the three sections. She advises you that
Section I (Professor X's section) appears to contain a
disproportionate number of students from the same private school in
Toronto, compared to sections II and III which contain a broader
representation of students overall.

\textbf{Question 1.2}: Conduct an ANCOVA in which you take into
account the high-school grades of students. Based on the ANCOVA
results, what can you conclude about the final exam scores from
Sections I, II and III of Calculus 101?  What can you conclude about
Professor X's claim? Will you give him a raise?

Include in your (brief) report and discuss: raw means; adjusted means; ANOVA tables; helpful graphical displays.


\begin{table}[ht]
  \begin{center}
    \begin{tabular}{|l|ccc|}
      \hline
      Calculus Section &I &II &III\\
      \hline
      Mean Calculus Grade &81.9 &75.3 &75.7\\
      Mean High School Math Grade &76.1 &67.5 &68.6\\
      \hline
    \end{tabular}
    \caption{Final Calculus Grades}
    \label{calctable}
  \end{center}
\end{table}


\end{document}

%%% Local Variables:
%%% mode: latex
%%% TeX-master: t
%%% End:
