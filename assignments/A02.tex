\documentclass[10pt]{article}

\usepackage{graphicx}
\usepackage{amssymb}
\usepackage{amsmath}
\usepackage{ifthen}
\usepackage{epsfig}
\usepackage{fancyvrb}
\usepackage{url}
\usepackage{fancyhdr}

\usepackage{amsmath}
\usepackage{mathspec}
\usepackage{microtype}
\setmainfont[
    Numbers={Monospaced,OldStyle},
    BoldFont={Guardian TextEgyp Medium},
  ]{Guardian TextEgyp}\linespread{1.2}
\setsansfont[BoldFont={Helvetica Neue Medium}]{Helvetica Neue}\linespread{1.2}
\setmonofont[Scale=0.90]{Menlo}\linespread{1.2}
\setmathrm{Palatino}
\setmathfont(Digits,Latin,Greek){Palatino}

\tolerance=5000

\usepackage[margin=1.25in,includeheadfoot]{geometry}

\setlength{\skip\footins}{10mm}      % obsessing about footnote spacing
\setlength{\parskip}{1ex}            % paragraph spacing
\setlength{\parindent}{0ex}          % paragraph indentation

\usepackage[compact,noindentafter]{titlesec}
\titleformat*{\section}{\sffamily\large\bfseries}
\titlespacing{\section}{0pt}{2ex}{1ex}
\titleformat*{\subsection}{\sffamily\normalsize\bfseries}
\titlespacing{\subsection}{0pt}{2ex}{0ex}

% LISTS
\usepackage{enumitem}
\setenumerate{nolistsep,itemsep=1.0ex,parsep=0.0ex,leftmargin=*}
\setitemize{nolistsep,itemsep=1.0ex,parsep=0.0ex,leftmargin=*}

\usepackage[iso]{datetime}
\usdate

% ------- CUSTOM TITLE FORMAT -------
%
\makeatletter
\renewcommand{\maketitle}{
\begin{flushleft}          % right align
{\Large\sffamily\bfseries\@title}   % increase the font size of the title
\vspace{3ex}\\            % vertical space between the title and author name
{\normalsize\sffamily\@author}           % author name
\vspace{0ex}\\             % vertical space between author name and date
\normalsize\sffamily\@date                     % date
\vspace{5ex}              % vertical space between the author block and abstract
\end{flushleft}
}
% -----------------------------------

% DEFINE SOME COLORS
\usepackage{color}
\definecolor{light-blue}{RGB}{37,128,162}
\definecolor{light-gray}{gray}{0.975}
\definecolor{dark-gray}{gray}{0.50}

% CODE LISTINGS
\usepackage{listings}
\lstset{columns=fullflexible,
  basicstyle=\ttfamily\small,
  backgroundcolor=\color{light-gray},
  frame=lines, frame=lrtb, framesep=3pt, rulecolor=\color{dark-gray},
  aboveskip=10pt,
  belowskip=15pt,
  breaklines=true,
  showstringspaces=false,
  keepspaces=true,
}


\title{Assignment 2---Introduction to Statistics Using R}
\author{Paul Gribble}
\date{Winter, 2017}


\begin{document}

\maketitle

\thispagestyle{empty}

{\flushleft \sffamily * Due Sunday February 5}

\section{Single Factor Between Subjects ANOVA}

The .csv file called \texttt{migraine.csv} contains 27 pain ratings
collected from migraine sufferers following the ingestion of one of
three drugs after the onset of migraine (drug A, B or C). Pain ratings
are on a scale of 1 (least pain) to 10 (most pain).

Load the data into \texttt{R} with the following code:

\begin{lstlisting}[language=R]
fname <- "http://www.gribblelab.org/stats/data/migraine.csv"
migraine <- read.table(fname, sep=",", header=TRUE)
\end{lstlisting}

\begin{enumerate}

\item Generate some kind of graphical display of the data. It's up to
  you what kind. Label the axes and give the Figure a title.

\item Perform a one-way analysis of variance to determine if there is
  evidence that pain is affected by drug. You can use the following
  commands in \texttt{R} if you like:

\begin{lstlisting}[language=R]
results <- aov(pain ~ drug, data=migraine)
summary(results)
\end{lstlisting}

\item Where appropriate, test the assumptions of ANOVA. You can use
  \texttt{bartlett.test} to test the homogeneity of variance
  assumption. You can use the \texttt{shapiro.test} to test the
  normality assumption. You can do the test either on all the data
  lumped together, or on each group separately---your choice.

\item With reference to the appropriate elements of the ANOVA output,
  what is your interpretation of the analysis?

\item The omnibus test is significant (\texttt{F(2,24)=11.91,
    p<0.000256}). Now test which groups are different from each
  other. Assume you haven't planned comparisons in advance. Conduct
  all possible pairwise tests, and be sure to apply an adjustment for
  Type-I error. Explain why you chose your particular adjustment.

  Hint: the \texttt{TukeyHSD} function in \texttt{R} can be applied to
  the output of the \texttt{aov} function (in the above code, the
  variable called \texttt{results}).

\item On the basis of your analyses? What can you conclude about each
  drug's affect on migraine pain?

\end{enumerate}


\end{document}


%%% Local Variables:
%%% mode: latex
%%% TeX-master: t
%%% End:
