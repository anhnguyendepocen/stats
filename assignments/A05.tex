\documentclass[10pt]{article}

\usepackage{graphicx}
\usepackage{amssymb}
\usepackage{amsmath}
\usepackage{ifthen}
\usepackage{epsfig}
\usepackage{fancyvrb}
\usepackage{url}
\usepackage{fancyhdr}

\usepackage{amsmath}
\usepackage{mathspec}
\usepackage{microtype}
\setmainfont[
    Numbers={Monospaced,OldStyle},
    BoldFont={Guardian TextEgyp Medium},
  ]{Guardian TextEgyp}\linespread{1.2}
\setsansfont[BoldFont={Helvetica Neue Medium}]{Helvetica Neue}\linespread{1.2}
\setmonofont[Scale=0.90]{Menlo}\linespread{1.2}
\setmathrm{Palatino}
\setmathfont(Digits,Latin,Greek){Palatino}

\tolerance=5000

\usepackage[margin=1.25in,includeheadfoot]{geometry}

\setlength{\skip\footins}{10mm}      % obsessing about footnote spacing
\setlength{\parskip}{1ex}            % paragraph spacing
\setlength{\parindent}{0ex}          % paragraph indentation

\usepackage[compact,noindentafter]{titlesec}
\titleformat*{\section}{\sffamily\large\bfseries}
\titlespacing{\section}{0pt}{2ex}{1ex}
\titleformat*{\subsection}{\sffamily\normalsize\bfseries}
\titlespacing{\subsection}{0pt}{2ex}{0ex}

% LISTS
\usepackage{enumitem}
\setenumerate{nolistsep,itemsep=1.0ex,parsep=0.0ex,leftmargin=*}
\setitemize{nolistsep,itemsep=1.0ex,parsep=0.0ex,leftmargin=*}

\usepackage[iso]{datetime}
\usdate

% ------- CUSTOM TITLE FORMAT -------
%
\makeatletter
\renewcommand{\maketitle}{
\begin{flushleft}          % right align
{\Large\sffamily\bfseries\@title}   % increase the font size of the title
\vspace{3ex}\\            % vertical space between the title and author name
{\normalsize\sffamily\@author}           % author name
\vspace{0ex}\\             % vertical space between author name and date
\normalsize\sffamily\@date                     % date
\vspace{5ex}              % vertical space between the author block and abstract
\end{flushleft}
}
% -----------------------------------

% DEFINE SOME COLORS
\usepackage{color}
\definecolor{light-blue}{RGB}{37,128,162}
\definecolor{light-gray}{gray}{0.975}
\definecolor{dark-gray}{gray}{0.50}

% CODE LISTINGS
\usepackage{listings}
\lstset{columns=fullflexible,
  basicstyle=\ttfamily\small,
  backgroundcolor=\color{light-gray},
  frame=lines, frame=lrtb, framesep=3pt, rulecolor=\color{dark-gray},
  aboveskip=10pt,
  belowskip=15pt,
  breaklines=true,
  showstringspaces=false,
  keepspaces=true,
}


\title{Assignment 5---Introduction to Statistics Using R}
\author{Paul Gribble}
\date{Winter, 2017}


\begin{document}

\maketitle

\thispagestyle{empty}

{\flushleft \sffamily * Due Friday Mar 24}

\section{Multiple Regression}

You are being considered for a position as the new coach of a
professional basketball team. You just finished a meeting with the
owners, who are interested in replacing the existing coach because the
team’s performance has been going downhill. The owners of the team
have attributed this downward trend to poor recruitment of new
players, who have been poor performers. The owners tell you that their
analysis of the data from the previous seasons indicates that one
particularly weak area has been ``field goals'' (when a player
attempts to score a basket from the 3-point line or beyond). The
team’s players, in particular the new recruits, tend to be poor at
field goals. The problem is, as the owners tell you, that data are not
available on field goal percentages, for players who are candidate
recruits; so when it comes time to recruit new players there is no
easy way to pick new recruits who are good field goal shooters. The
owners tell you that they have been collecting data for years and
years on the new recruits and their field goal percentages, and a
series of other measures that are available for the new potential
recruits as well. She suggests that ``if you are smart enough to use
the historical data to be able to predict who among the potential new
recruits will be the best field goal shooters, you are hired!''

In the file called \texttt{bball.csv}\footnote{\texttt{http://www.gribblelab.org/stats/data/bball.csv}} you will find historical data on 105 previous
recruits. The variables are:

\begin{center}
\begin{tabular}{c|l}
	\texttt{GAMES} &\# games played in previous season \\
	\texttt{PPM}   &average points scored per minute \\
	\texttt{MPG}   &average minutes played per game \\
	\texttt{HGT}   &height of player (centimetres) \\
	\texttt{FGP}   &field-goal percentage (\% successful shots from 3-point line) \\
	\texttt{AGE}   &age of player (years) \\
	\texttt{FTP}   &percentage of successful penalty free throws \\	
\end{tabular}
\end{center}

In the file \texttt{bball2.csv}\footnote{\texttt{http://www.gribblelab.org/stats/data/bball2.csv}} you will find data on 5 potential new recruits.

\begin{enumerate}

\item Use multiple regression to develop a model based on the data in
  \texttt{bball.csv} that will allow you to predict the field goal
  percentages for the 5 new recruits.

\item Which player will you suggest to the owners that they hire?
  Explain your decision.

\item What is the precision (in units of \texttt{FGP}) with which you
  can predict field goal percentage?

\item After giving them your recommendation and explaining your model,
  the owners ask you ``does the player's age have anything to do with
  this?''

\item ``What about the player's height? Would I be better off taking
  on taller players?''

\item ``How many more (or fewer) field goals would a player score if
  he was 3 inches taller?''

\end{enumerate}



\end{document}


%%% Local Variables:
%%% mode: latex
%%% TeX-master: t
%%% End:
